\section{总结}
\label{sec:conclusion}

本报告中,我阐述了疫情环境下低风险旅行规划模拟系统\textsc{TravelAgency}的背景、目的、主要功能、具体设计、开发过程等。\textsc{TravelAgency}是一个实用高效、简洁美观的旅行规划系统。能够基于两种不同的旅行策略(最小风险、限时最小风险)进行旅行方案的规划,并且能实时地显示出各个旅客的旅程状态。这一系统由三大模块组成:前端界面、服务器、算法核心。采用了丰富的技术栈,进行了高效的开发,在算法核心中使用了图、链表、可持久化链表、队列等数据结构。并应用SPFA算法,在隐含的时间-城市图、时长-城市图上求解最优路径。

\textsc{TravelAgency}系统具有如下优点:
\begin{enumerate}[(a)]
  \item \textbf{功能齐全。} 系统实现了所有要求的功能,以及两个选做功能。能够读取、管理城市线路信息,进行两种策略的路径规划;能够实时显示各个旅客的旅行状态,且在规划的过程中,能够考虑到旅行途中带来的风险。
  \item \textbf{界面美观,简洁明了。} 系统利用前端技术实现了用户图形界面。调用了高德地图API,采用了Material Design风格的界面,美观大方,简洁明了,且能对需要的信息进行高效、清晰的展示。
  \item \textbf{采用了高效的数据结构与算法。} 系统采用了高效的数据结构与算法。利用邻接链表来表示城市线路图中的边;利用可持久化链表来表示旅程。并且使用了SPFA来进行旅行方案的规划。其平均时间复杂度可达到$\mathcal O(|\mathcal E^\prime|)$,最坏时间复杂度为$\mathcal O(|\mathcal V^\prime| |\mathcal E^\prime|)$。
  \item \textbf{模块划分明确,开发效率高。} 系统被划分为三个主要模块,使用不同的技术进行开发。用户界面由前端技术完成,效率高,界面美观;服务器由Python开发,Python具有成熟的WebSocket工具库与并发库,开发效率高;算法核心由C++进行开发,能够实现很高的时间、空间效率,且为编译型语言,运行效率高,保证核心算法的运行速度。因而,各大模块各司其职,且采取了合适的技术进行开发,兼顾了开发效率与系统质量。
\end{enumerate}

而另一方面,\textsc{TravelAgency}系统还有一些需要改进的不足之处:(1) 在功能方面,可以添加更多、更完善的功能,例如:添加用户的注册机制,让每一用户可以同时规划多个旅程,并只显示自己规划的旅程;在旅程进行过程中,增加修改目的地的功能,提高系统灵活性;考虑用户更加细致的需求,如不愿意乘坐飞机,以及追求时间最短等。(2) 在前端界面方面,可以考虑使用WebAssembly这样的新技术,来获得更高的效率。(3) 在服务器与后端接口方面,现在使用的是 \lstinline{ctypes} 对动态库进行调用,需要动态地设置需要调用的函数的参数、返回值信息,容易出错,难以维护,可以替换成 \lstinline{cffi} 在编译期进行库配置,具有更高的开发效率。

综上所述,\textsc{TravelAgency}是一个高效美观、设计良好的低风险旅行模拟规划系统。系统具有诸多优点,也有一些需要改进之处。设计、实现这一课程设计项目的过程让我获益良多,在之后的软件开发过程中,我将吸收其中的经验,更进一步。
