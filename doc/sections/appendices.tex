\section{工程的编译与运行}
\label{sec:compile-and-run}

\subsection{环境要求}

以下是编译、运行所需要的工具要求:
\begin{itemize}
  \item Chrome 83+
  \item Python 3.7+
  \item CMake 3.17
  \item Ninja 1.10
\end{itemize}

Python依赖的第三方包如下:
\begin{enumerate}
  \item websockets 8.1
\end{enumerate}

项目的开发、测试环境为 macOS Catalina 10.15.5。\textbf{注意:在\texttt{./py/\_travel\_agency.py}文件中,硬编码了动态链接库的位置。在macOS下动态链接库的后缀名为\texttt{.dylib},如果在Windows下或者Linux下运行代码,要在编译动态库后将动态库文件名相应修改。}

\subsection{编译方法}

本项目中,C++算法核心部分使用CMake + Ninja工具链进行编译。前端界面与Python服务器均无需编译。C++动态库的编译过程如下。

首先,调用CMake生成编译文件。在项目根目录执行下列指令。
\begin{verbatim}
mkdir -p build/default
cd build/default
cmake -GNinja -DBUILD_SHARED_LIBS=ON ../..
\end{verbatim}

随后,调用Ninja进行编译。同样在项目根目录执行下列指令。
\begin{verbatim}
cd build/default
ninja
\end{verbatim}

Ninja将打印出编译的过程与结果。若编译顺利,\texttt{./build/default/} 文件夹内将生成动态库文件。

\textbf{注意:如果在Windows下编译,若发现编译产生的DLL文件出现了乱码现象,则需要在 \texttt{./src/travel\_agency.hh}文件中的函数定义中添加 \lstinline{\_\_declspec(dllexport)}宏。在类Unix环境下,只需要将函数定义包裹在 \lstinline{extern "C"} 中,编译器就不会对导出的函数名进行修饰,而在Windows下,添加上述宏是必要的。由于本项目在 macOS 下开发,因而没有添加这些宏。如果着手自己编译,需要注意这一点。}

\subsection{运行方法(预编译的Windows可执行文件)}

提交的文件中已经包含了预编译的Windows可执行文件。只需要双击运行,\textbf{并且根据给出的提示在浏览器中打开相应网页}即可以运行系统。

\subsection{运行方法(自行编译)}

在运行之前,请保证:
\begin{enumerate}
  \item 动态链接库已经正确编译,产生在\texttt{./build/default/}文件夹内。在Windows下,为\texttt{.dll}后缀的文件,在Linux下,为\texttt{.so},在macOS下,为\texttt{.dylib};
  \item 如果在Windows下或Linux下运行程序,需要根据编译生成的动态库路径,修改\texttt{./py/\_travel\_agency.py}文件中动态库的位置。
  \item 如果在Windows下运行程序,需要确保Python版本为3.7。若产生DLL加载错误,则很有可能是依赖的DLL没有放在Python的DLL加载路径中,如 \lstinline{libstdc++}等等。需要找寻到对应的DLL并放入加载目录。
\end{enumerate}

首先,运行Python服务器。
\begin{verbatim}
cd py
python main.py
\end{verbatim}

然后,在浏览器(推荐使用新版本的Chrome)中打开\texttt{./gui/index.html}文件。这可以通过打开链接:\texttt{file:///\$PROJECT\_PATH/gui/index.html}来实现。其中\texttt{\$PROJECT\_PATH}为项目文件夹所在路径。网页界面将自动连接服务器。系统开始运行。

\section{系统使用说明}
\label{sec:manual}

系统界面分为两个部分:地图与信息面板。地图上能够显示城市信息、旅程详细信息与状态等。信息面板则呈现各类信息,包括事件、旅行列表,也有添加按钮。接下来分点详细阐释系统功能与使用方法。

\textbf{查看城市信息。} \quad 在界面的地图部分上,用圆圈标出了系统支持的旅行城市。圆圈上方为城市名称。圆圈的颜色代表了城市的风险等级:\textit{{\color{high-risk}红色}}代表高风险地区;\textit{{\color{mid-risk}黄色}}代表中风险地区;\textit{{\color{low-risk}绿色}}代表低风险地区。

\textbf{查看时间信息、控制时间运行。} \quad 信息面板的上方给出了系统的当前时间。系统时间从第 0 天 0 时开始,以固定的间隔,以1小时为单位向前推进。下面的按钮能够暂停、重启系统时间的运行。

\definecolor{air}{HTML}{a0c4ff}
\definecolor{subway}{HTML}{457b9d}
\definecolor{highway}{HTML}{1d3557}

\textbf{旅行列表、查看旅行详情。} \quad 信息面板时间的下方,为旅行列表。旅行列表中的每一项都是正在进行中的旅程。列表中每一项都给出了旅程的简要信息,包括:旅客编号、出发时间、出发地点、到达时间、目的地点、旅行时长、旅行风险。点击列表中的某一项,将显示该旅程的详细信息,包括每一步搭乘的线路信息,线路时长,在线路中产生的风险等。同时,地图中将绘制出该条线路的详细路径。并且以灰色圆形指示旅客当前所处位置。绘制的路径中,\textit{\color{air}天蓝色}代表飞机,\textit{\color{subway}青色}代表火车,\textit{\color{highway}靛色}代表公路。将鼠标移到旅客标识上,将显示旅客状态。

\textbf{添加旅程。} \quad 点击右下角按钮,将打开添加旅程窗口,窗口打开后,将自动暂停系统时间。在窗口中需要输入旅程信息、选择旅行策略。当所有信息都被输入且合法时,将显示旅行预览,更新信息,旅行预览也会更新。当信息有问题(出发地、目的地相同,或找不到可行的方案)时,旅行预览不会显示。点击确认按钮,若信息无问题,系统将旅行添加到列表内;若信息有问题,将产生提示。点击取消按钮,将返回到主界面,输入的信息将被丢弃。

\textbf{查看系统日志。} \quad 在Python服务器终端内,可以看到系统日志。系统日志将打印出客户端的连接、断开信息;客户端的请求、服务器的响应信息;运行规划算法的信息与耗时;其他提示信息等。

\section{性能测试}
\label{sec:perf-test}

\begin{table}[t]
  \centering
  \resizebox{\textwidth}{!}{%
  \begin{tabular}{c|cccccc}
    \toprule
    项目 & 平均用时 (ms) & 最大用时 (ms) & 最小用时 (ms) & 平均空间 (MiB) & 最大空间 (MiB) & 最小空间 (MiB) \\
    \midrule
最小风险 & 18.10 & 29.84 & 10.44 & 70.39 & 71.53 & 64.89 \\
限时最小风险 & 33.42 & 58.50 & 8.58 & 115.06 & 129.33 & 71.38 \\
全部 & 25.76 & 58.50 & 8.58 & 92.73 & 129.33 & 64.89 \\
    \bottomrule
  \end{tabular}%
  }
  \caption{在随机生成的$10^4$座城市,$10^5$条线路的地图上的性能测试表现}
  \label{tab:perf-test}
\end{table}

为了进一步验证系统效率,我随机生成了大规模的城市图,在上面运行系统的规划算法,测量不同策略的规划时间与内存消耗。具体地,将城市数设为$10^4$,并随机生成$10^5$条线路,每种策略运行100次,性能测试的结果在表 \ref{tab:perf-test} 中给出。从中可以看出,在这一大规模的图上进行规划,算法仍然能够以非常快地速度给出结果,证明了系统中算法的高效性。



























