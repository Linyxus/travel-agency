\section{引言}

在本文中,我将介绍\textsc{TravelAgency}:新冠疫情环境下的低风险旅行规划模拟系统。该系统能够根据各个城市的风险等级,以及城市之间的线路信息,基于两种不同的策略,为用户模拟规划出低风险的旅行方式。

系统实现了所有必须完成的需求,包含基于两种策略的规划:最小风险旅程与带时间约束的最小风险旅程(也即限时最小风险)。并且实现了所有选作请求,能够在地图上实时绘制出旅程的状态,并且考虑到了旅行线路带来的风险。系统使用了高效的数据结构,利用图来描述城市线路信息,利用邻接表来存储图,利用可持久化链表来描述旅程;使用SPFA作为两种策略的规划算法,通过对系统的性能测试展现了系统的效率。

系统由多种语言完成开发,大抵可分为三个部分。(a) 图形用户界面,由前端技术实现,由 HTML、CSS和JavaScript进行开发。采用Vus.js作为前端框架,并且调用了高德地图API,实现绘制城市信息,实时显示旅行线路与状态;(b) 算法核心,由C++进行开发,采用了图作为城市线路的数据结构,利用可持久化链表作为描述旅程的数据结构,实现了较高的空间效率。且应用SPFA实现了高效的线路规划;(c) 服务器。前端技术实现的用户界面需要和C++实现的高效算法逻辑进行通信,为了实现他们之间高效、简单的通信,使用Python开发了WebSocket服务器。一方面,服务器使用WebSocket与前端用户界面通信,另一方面,利用 \lstinline{ctypes} 库调用C++算法核心的动态库。起到了桥接作用。

\begin{figure}[h]
\centering
\includegraphics[width=\textwidth]{figures/language_arch}
\caption{系统架构}
\label{fig:language-arch}
\end{figure}

本文将分几个部分介绍\textsc{TravelAgency}旅行规划模拟系统的各方面。在第 \ref{sec:preliminary} 节中,将介绍\textsc{TravelAgency}的背景,需求功能分析与设计概要。在第 \ref{sec:showcase} 节中,将展示系统功能,解说系统的基本特性。在第 \ref{sec:design} 节中,将细致地讲解系统的设计,包含数据结构、算法,以及前端、服务器的设计概述。在第 \ref{sec:experiments} 节中,将讲述样例的基本信息以及运行结果分析。在第 \ref{sec:conclusion} 节中,将进行总结讨论,指出系统的优点与改进之处。除此以外,在附录 \ref{sec:compile-and-run} 中详细介绍了运行、编译工程的方法。在附录 \ref{sec:manual} 中给出了系统的用户使用说明。在附录 \ref{sec:perf-test} 展示了系统性能测试的结果。