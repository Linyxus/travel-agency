\section{背景}
\label{sec:preliminary}

\paragraph{系统背景}

\textsc{TravelAgency}的背景是在新冠疫情环境下,对低风险旅行的规划与模拟。新冠疫情以来,全国上下团结一心,共同抵抗疫情,然而也未出行带来的诸多不便。如今疫情已经受到控制,然而出行仍然具有一定风险。\textsc{TravelAgency}旨在根据城市风险等级、线路信息等条件,为用户规划出合理的、有或没有时间限制的最低风险路线。

\paragraph{城市风险等级与等待风险}

系统假设城市风险分为三级:高风险、中风险、低风险。且具有各自的风险系数,如表 \ref{tab:city-level-risk} 所示。
\begin{table}[h]
\centering
\begin{tabular}{cccc}
\toprule
风险等级 & 低 & 中 & 高 \\
\midrule
风险系数 & 0.2 & 0.3 & 0.9 \\
\bottomrule
\end{tabular}
\caption{城市风险系数表}
\label{tab:city-level-risk}
\end{table}

假设某一旅客在$t_a$到达城市$c$,城市$c$具有风险系数$r_c$,且他下面要进行的旅行线路在$t_d$出发。则因等待而产生的风险定义为
\begin{equation}
  R_w \triangleq \| t_d - t_a \| \cdot r_c,
\end{equation}
其中,$\| t_d - t_a \| \triangleq (t_d - t_a + 24) \% 24$为需要等待的时间。

\paragraph{线路信息与路程风险}

系统假设一条线路由如下信息定义:出发地点、目的地点、出发时间、所用时长、类型。系统假设所有系统都以一天为周期重复,因而任一线路的出发时间$t_d$满足$t_d = 0, 1, 2, \cdots, 23$。线路时长没有限制。线路类型有三种:飞机、火车、客车。不同类型的线路具有不同的风险系数,如表 \ref{tab:line-type-risk} 所示。
\begin{table}[h]
\centering
\begin{tabular}{cccc}
\toprule
线路类型 & 客车 & 火车 & 飞机 \\
\midrule
风险系数 & 2 & 3 & 9 \\
\bottomrule
\end{tabular}
\caption{线路风险系数表}
\label{tab:line-type-risk}
\end{table}

而假设某一线路$l$从城市$c$出发,时长为$\tau$。$l$和$c$分别具有风险系数$r_l$与$r_c$。则此条线路带来的路程风险定义为
\begin{equation}
  R_t \triangleq r_l \tau r_c.
\end{equation}

\paragraph{城市地图}

将城市信息与线路信息建模为一张图,定义$\mathcal G = (\mathcal V, \mathcal E, r_c, r_l, t_d, t_a, \tau, v_s, v_d)$。其中$\mathcal V = \{ c_1, c_2, \cdots, c_n \}$为城市集合,$\mathcal E = \{ l_1, l_2, \cdots, l_k \}$为线路集合。$r_c : \mathcal V \rightarrow \mathds R$为城市到其风险系数的映射。$r_l : \mathcal E \rightarrow \mathds R$为线路到其风险系数的映射。$t_d, t_a : \mathcal E \rightarrow \{ 0, 1, \cdots, 23 \}$为线路到其出发、到达时间的映射。$\tau : \mathcal E \rightarrow \mathds N_+$为线路到其时长的映射。而$v_s, v_d : \mathcal E \rightarrow \mathcal V$为线路到其出发、目的城市的映射。可以知道,一般来说这张图是有向多重图。

\paragraph{旅行规划与两种策略}

系统需要解决的是低风险旅行规划问题。假设某一旅客在$t_0$时从城市$c_0$出发,希望到达城市$c$。系统需要模拟规划出旅程,经过线路$l_0, l_1, l_2, \cdots, l_k$。可以知道这些线路满足:$v_d(l_i) = v_s(l_{i+1}), i = 0, 1, \cdots, k-1$,且$v_s(l_0) = c_0$,$v_d(l_k) = c$。而总风险可以写为
\begin{equation}
  R(l_0, l_1, \cdots, l_k) = \sum_{i=0}^k \left[ \tau_i r_c(c_i) + r_l(l_i) \tau(l_i) r_c(c_i) \right],
\end{equation}
其中,$\tau_i$为等待时间,有$\tau_0 = \| t_d(l_0) - t_0 \|, \tau_i = \| t_d(l_i) - t_a(l_{i-1}) \|, i = 1, 2, \cdots, k-1$。而$c_i = v_s(l_i), i = 0, 1, \cdots, k$为出发城市。系统需要找到符合条件的线路列表,使得$R$最小,也即如下的最优化问题
\begin{equation}
  \begin{aligned}
    \min_{l_0, \cdots, l_k} \quad & R(l_0, l_1, \cdots, l_k) \\
    \text{s. t.} \quad & v_d(l_i) = v_s(l_{i+1}), i = 0, 1, \cdots, k-1 \\
    & v_s(l_0) = c_0, v_d(l_k) = c \\
  \end{aligned}
\end{equation}
而对于带有时间约束的最小风险策略,旅途总时长可以类似地写为
\begin{equation}
  T(l_0, l_1, \cdots, l_k) = \sum_{i=0}^k \left[ \tau_i + \tau(l_i) \right].
\end{equation}
假设有时间约束$T \le \Gamma$,则限时最小风险策略可以被认为是求解下列最优化问题
\begin{equation}
  \begin{aligned}
    \min_{l_0, \cdots, l_k} \quad & R(l_0, l_1, \cdots, l_k) \\
    \text{s. t.} \quad & v_d(l_i) = v_s(l_{i+1}), i = 0, 1, \cdots, k-1 \\
    & v_s(l_0) = c_0, v_d(l_k) = c \\
    & T(l_0, l_1, \cdots, l_k) \le \Gamma \\
  \end{aligned}
\end{equation}


